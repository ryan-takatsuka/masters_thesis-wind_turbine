This thesis develops a model of the Cal Poly Wind Turbine that is used to determine if there is an imbalance in the turbine rotor.  A theoretical model is derived to estimate the expected vibrations when there is an imbalance in the rotor.  Vibration and acceleration data are collected from the turbine tower during operation to confirm the model is useful and accurate for determining imbalances in the turbine.  

Digital signal processing techniques for analyzing the vibration data are explored and tested with simulation data.  This includes frequency shifts, lock-in amplifiers, phase-locked loops, discrete Fourier transforms, and decimation filters.  The processed data is fed into an algorithm that determines if there is an imbalance.

The detection algorithm consists of a machine learning classification model that uses experimental data to train and increase the success rate of the imbalance detection.  Various models are explored, including the K-Nearest Neighbors algorithm, logistic regression, and neural networks.  These models have trade-offs between mathematical complexity, required computing power, scalability, and accuracy.  With proper implementations of these detection models, the imbalance detection accuracy was measured to be about 90\%.